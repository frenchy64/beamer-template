% $Header$

\documentclass[aspectratio=169]{beamer}
\usepackage{verbatim}
\makeatletter
\newcommand{\verbatimfont}[1]{\def\verbatim@font{#1}}%
\makeatother

% This file is a solution template for:

% - Talk at a conference/colloquium.
% - Talk length is about 20min.
% - Style is ornate.

\usepackage{graphicx}

% Copyright 2004 by Till Tantau <tantau@users.sourceforge.net>.
%
% In principle, this file can be redistributed and/or modified under
% the terms of the GNU Public License, version 2.
%
% However, this file is supposed to be a template to be modified
% for your own needs. For this reason, if you use this file as a
% template and not specifically distribute it as part of a another
% package/program, I grant the extra permission to freely copy and
% modify this file as you see fit and even to delete this copyright
% notice. 

\newtheorem{answeredquestions}[theorem]{Answered Questions}
\newtheorem{openquestions}[theorem]{Open Questions}

\mode<presentation>
{
  %\usetheme{Warsaw}
  \usetheme{default}
  % or ...

  \setbeamercovered{transparent}
  % or whatever (possibly just delete it)
}


\usepackage[english]{babel}
% or whatever

%\usepackage[latin1]{inputenc}
% or whatever

\usepackage{times}
\usepackage{fontspec}
\usepackage[T1]{fontenc}
% title font
\setmainfont{Chaparral Pro}
% verbatim font
\setmonofont{Anonymous Pro}
% everything-else font
\setsansfont{Chaparral Pro}

% Or whatever. Note that the encoding and the font should match. If T1
% does not look nice, try deleting the line with the fontenc.
%\usefonttheme[onlylarge]{Anonymous Pro}
%\usefonttheme[onlysmall]{structurebold}

\title%[Short Paper Title] % (optional, use only with long paper titles)
{Practical Optional Types \\for Clojure}

\setbeamerfont{title}{family=\rmfamily,series=\mdseries,size={\fontsize{28}{36}}}
\setbeamercolor{title}{fg=black}
%\subtitle
%{Include Only If Paper Has a Subtitle}

\author%[Author, Another] % (optional, use only with lots of authors)
{Ambrose Bonnaire-Sergeant%~\inst{1}
}
% - Give the names in the same order as the appear in the paper.
% - Use the \inst{?} command only if the authors have different
%   affiliation.

\institute[Indiana University] % (optional, but mostly needed)
{
  %\inst{1}%
  Department of Computer Science\\
  Indiana University
  }
% - Use the \inst command only if there are several affiliations.
% - Keep it simple, no one is interested in your street address.

\date%[] % (optional, should be abbreviation of conference name)
{}
% - Either use conference name or its abbreviation.
% - Not really informative to the audience, more for people (including
%   yourself) who are reading the slides online

%\subject{Theoretical Computer Science}
% This is only inserted into the PDF information catalog. Can be left
% out. 



% If you have a file called "university-logo-filename.xxx", where xxx
% is a graphic format that can be processed by latex or pdflatex,
% resp., then you can add a logo as follows:

% \pgfdeclareimage[height=0.5cm]{university-logo}{university-logo-filename}
% \logo{\pgfuseimage{iu-logo}}



% Delete this, if you do not want the table of contents to pop up at
% the beginning of each subsection:
%\AtBeginSubsection[]
%{
%  \begin{frame}<beamer>{Outline}
%    \tableofcontents[currentsection,currentsubsection]
%  \end{frame}
%}


% If you wish to uncover everything in a step-wise fashion, uncomment
% the following command: 

%\beamerdefaultoverlayspecification{<+->}


\begin{document}
{
% no navigation bar
\beamertemplatenavigationsymbolsempty

\begin{frame}
  \titlepage
\end{frame}

\begin{frame}{Outline}
  \tableofcontents[pausesections]
  % You might wish to add the option [pausesections]
\end{frame}


% Structuring a talk is a difficult task and the following structure
% may not be suitable. Here are some rules that apply for this
% solution: 

% - Exactly two or three sections (other than the summary).
% - At *most* three subsections per section.
% - Talk about 30s to 2min per frame. So there should be between about
%   15 and 30 frames, all told.

% - A conference audience is likely to know very little of what you
%   are going to talk about. So *simplify*!
% - In a 20min talk, getting the main ideas across is hard
%   enough. Leave out details, even if it means being less precise than
%   you think necessary.
% - If you omit details that are vital to the proof/implementation,
%   just say so once. Everybody will be happy with that.

\section{Motivation}

\subsection{The Basic Problem That We Studied}

\begin{frame}{Make Titles Informative. Use Uppercase Letters.}{Subtitles are optional.}
  % - A title should summarize the slide in an understandable fashion
  %   for anyone how does not follow everything on the slide itself.

  \begin{itemize}
  \item
    Use \texttt{itemize} a lot.
  \item
    Use very short sentences or short phrases.
  \end{itemize}
\end{frame}

\begin{frame}{Make Titles Informative.}

  You can create overlays\dots
  \begin{itemize}
  \item using the \texttt{pause} command:
    \begin{itemize}
    \item
      First item.
      \pause
    \item    
      Second item.
    \end{itemize}
  \item
    using overlay specifications:
    \begin{itemize}
    \item<3->
      First item.
    \item<4->
      Second item.
    \end{itemize}
  \item
    using the general \texttt{uncover} command:
    \begin{itemize}
      \uncover<5->{\item
        First item.}
      \uncover<6->{\item
        Second item.}
    \end{itemize}
  \end{itemize}
\end{frame}


\subsection{Previous Work}
\begin{frame}
  \frametitle{What Are Prime Numbers?}
  \begin{definition}
    A \alert{prime number} is a number that has exactly two divisors.
  \end{definition}
\pause
	\begin{example}
		\begin{itemize}
			\item 2 is prime (two divisors: 1 and 2).
\pause
			\item 3 is prime (two divisors: 1 and 3).
\pause
			\item 4 is not prime (\alert{three} divisors: 1, 2, and 4).
		\end{itemize}
	\end{example}
\end{frame}



\section{Our Results/Contribution}

\subsection{Main Results}

\begin{frame}[c]
  \frametitle{There Is No Largest Prime Number}
  \framesubtitle{The proof uses \textit{reductio ad absurdum}.}
  \begin{theorem}
    There is no largest prime number.
  \end{theorem}
\only<3->{
	\begin{proof}
		\begin{enumerate}
			\item<1-2> Suppose $p$ were the largest prime number.
			\item<2-> Let $q$ be the product of the first $p$ numbers.
			\item<3-> Then $q + 1$ is not divisible by any of them.
			\item<1-> But $q + 1$ is greater than $1$, thus divisible by some prime
				number not in the first $p$ numbers.\qedhere
		\end{enumerate}
	\end{proof}
}
  \only<2->{The proof used \textit{reductio ad absurdum}.}
\end{frame}

\begin{frame}
  \frametitle{What's Still To Do?}
  \begin{answeredquestions}
    How many primes are there?
  \end{answeredquestions}
  \begin{openquestions}
    Is every even number the sum of two primes?
  \end{openquestions}
\end{frame}

\begin{frame}
	\frametitle{What's Still To Do?}
\begin{columns}[t]
	\column{.5\textwidth}
  \begin{answeredquestions}
    How many primes are there?
  \end{answeredquestions}

\pause
	\column{.5\textwidth}
  \begin{openquestions}
    Is every even number the sum of two primes?\cite{Goldbach1742}
  \end{openquestions}
\end{columns}
\end{frame}

\begin{frame}[fragile]
  \frametitle{An Algorithm For Finding Prime Numbers.}
\begin{verbatim}
int main (void)
{
  std::vector<bool> is_prime (100, true);
  for (int i = 2; i < 100; i++)
    if (is_prime[i])
      {
        std::cout << i << " ";
        for (int j = i; j < 100; is_prime [j] = false, j+=i);
      }
return 0; }
\end{verbatim}
  \begin{uncoverenv}<2>
    Note the use of \verb|std::|.
  \end{uncoverenv}
\end{frame}


\begin{frame}[fragile]
  \frametitle{An Algorithm For Finding Primes Numbers.}
\begin{semiverbatim}
\uncover<1->{\alert<0>{int main (void)}}
\uncover<1->{\alert<0>{\{}}
\uncover<1->{\alert<1>{  \alert<4>{std::}vector<bool> is_prime (100, true);}}
\uncover<1->{\alert<1>{  for (int i = 2; i < 100; i++)}}
\uncover<2->{\alert<2>{ if (is_prime[i])}}
\uncover<2->{\alert<0>{ \{}}
\uncover<3->{\alert<3>{ \}}}
\uncover<3->{\alert<3>{ \alert<4>{std::}cout << i << " ";}}
\uncover<3->{\alert<3>{ for (int j = i; j < 100;}}
\uncover<2->{\alert<0>{ is_prime [j] = false, j+=i);}}
\uncover<1->{\alert<0>{  return 0;}}
\uncover<1->{\alert<0>{\}}}
\end{semiverbatim}
  \visible<4->{Note the use of \alert{\texttt{std::}}.}
\end{frame}


\subsection{Basic Ideas for Proofs/Implementation}

\begin{frame}{Make Titles Informative.}
\end{frame}

\begin{frame}{Make Titles Informative.}
\end{frame}

\begin{frame}{Make Titles Informative.}
\end{frame}



\section*{Summary}

\begin{frame}{Summary}

  % Keep the summary *very short*.
  \begin{itemize}
  \item
    The \alert{first main message} of your talk in one or two lines.
  \item
    The \alert{second main message} of your talk in one or two lines.
  \item
    Perhaps a \alert{third message}, but not more than that.
  \end{itemize}
  
  % The following outlook is optional.
  \vskip0pt plus.5fill
  \begin{itemize}
  \item
    Outlook
    \begin{itemize}
    \item
      Something you haven't solved.
    \item
      Something else you haven't solved.
    \end{itemize}
  \end{itemize}
\end{frame}



% All of the following is optional and typically not needed. 
\appendix
\section<presentation>*{\appendixname}
\subsection<presentation>*{For Further Reading}

\begin{frame}[allowframebreaks]
  \frametitle<presentation>{For Further Reading}
    
  \begin{thebibliography}{10}
    
  \beamertemplatebookbibitems
  % Start with overview books.

  \bibitem{Author1990}
    A.~Author.
    \newblock {\em Handbook of Everything}.
    \newblock Some Press, 1990.
 
    
  \beamertemplatearticlebibitems
  % Followed by interesting articles. Keep the list short. 

  \bibitem{Someone2000}
    S.~Someone.
    \newblock On this and that.
    \newblock {\em Journal of This and That}, 2(1):50--100,
    2000.
  \end{thebibliography}
\end{frame}

	\begin{thebibliography}{10}
		\bibitem{Goldbach1742}[Goldbach, 1742]
			Christian Goldbach.
			\newblock A problem we should try to solve before the ISPN '43 deadline,
			\newblock \emph{Letter to Leonhard Euler}, 1742.
	\end{thebibliography}

\end{document}


